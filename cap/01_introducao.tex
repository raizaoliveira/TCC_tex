\chapter{INTRODUÇÃO}
\label{chap:introducao}
 \thispagestyle{plain}


\quad Nos primeiros sistemas computacionais a comunicação entre pessoas e máquinas era realizada através de terminais por linha de comando.  Apenas especialistas conseguiam utilizar estes sistemas.
Depois, no início da década de 70, com a criação do mouse e a introdução da interface gráfica  os sistemas tornaram-se mais amigáveis ao usuário, podendo ser utilizados por pessoas comuns sem necessidade de conhecimento técnico. Com o passar dos anos a interação entre pessoas e máquinas tornou-se mais intuitiva com as diversas interfaces entre o usuário e o sistema. No fim da década de 70 iniciaram-se as pesquisas de reconhecimento de fala.
Interfaces por meio de fala são utilizadas em diversas áreas, tais como: sistemas embarcados, automação residencial, operações bancárias, conversão fala texto e dispositivos móveis.

\quad O reconhecimento da fala é um campo de estudo amplo e necessário as diversas tecnologias que utilizam
desta como um  meio de comunicação entre o usuário e o sistema. Utilizar a fala como entrada de um sistema
torna a comunicação entre o usuário e o sitema mais direta, intuitiva, rápida e precisa.Como um campo de ampla aplicacação, o reconhecimento de fala tem diversos projetos em diferentes partes do mundo. Dentre os quais se destaca o projeto CMU Sphinx da universidade americana Carnegie Mellon. O projeto já tem cerca de 20 anos de pesquisas na área de reconhecimento de fala e de voz. Trata-se de um projeto open source voltado para linux, mas também conta com uma versão em java multiplataforma. O CMU Sphinx oferece suporte para várias linguagens, dentre elas o inglês, alemão, russo, francês e espanhol. 
O reconhecimento de fala pode ser classificado de acordo com o tamanho do vocábulario, de acordo com os algoritmos utilizados e de acordo com o tipo de fala a ser reconhecida (contínua ou discreta).



%A introdução deve conter a delimitação do tema, o problema, a justificativa e o
%objetivo do projeto, que podem vir em subseções separadas ou não.
%É muito importante ressaltar que a delimitação do tema requer clareza a respeito do
%campo de conhecimento a que pertence o assunto. O problema é o objeto de pesquisa ou de
%estudo. Optou-se, neste exemplo, em separar em subseções a justificativa e o(s) objetivo(s).\\
%No caso de projeto de pesquisa, que esteja vinculado a um grupo de pesquisa
%institucional, neste item é necessário acrescentar a denominação do grupo, que esteja
%devidamente certificado pela Unifra, e a denominação da linha de pesquisa a que pertence o
%projeto

\section{Justificativa}
\label{sec:justificativa}



O reconhecimento de palavaras ditas é um campo de estudo de extrema importância para uma melhor comunicação entre usuários e sistema.
%Na justificativa mencionam-se a relevância científica do trabalho, a contribuição da
%pesquisa e que benefício poderá trazer à comunidade ou à sociedade. Ainda devem estar claros
%o motivo da escolha do tema e as possibilidades de realização da pesquisa.
\section{Objetivos}
\label{sec:objetivos}

O objetivo deste trabalho é estudar os principais métodos de reconhecimento de fala. Analisar os algoritmos utilizados, suas vantagens e desvantagens. Apresentar os resultados para um pequeno vocábulario.
%A definição dos objetivos determina o que se quer atingir com a realização do
%trabalho de pesquisa. Objetivo é sinônimo de meta, fim.
%Uma sugestão interessante, na redação dos objetivos, é utilizar, no início das
%sentenças, o verbo no infinitivo, tais como: esclarecer tal coisa, definir tal assunto, procurar
%aquilo, permitir algo, demonstrar alguma coisa, entre outros.
%Alguns autores separam os objetivos em objetivo geral e objetivos específicos, mas
%não há regra a ser cumprida quanto a isso. Caso se opte em separá-los, tem-se:
\subsection{Objetivo geral}
\label{subsec:objetivogeral}
Estudar e analisar os algoritmos existentes para o reconhecimento de palavras  ditas em um vocábulario pequeno e um ambiente não controlado.
%O objetivo geral vincula-se à própria significação geral do tema proposto pelo
%projeto, ou seja, significa traçar as principais metas que norteiam a pesquisa.
\subsection{Objetivo específico}
\label{subsec:objetivoespecifico}
%Descrever aqui o(s) propósito(s) específico(s) para atingir um ponto de vista do tema,
%um ângulo a ser pesquisado, permitindo atingir o objetivo geral. Aconselha-se, na redação
%desta seção, não ser prolixo.
Apontar a melhor solução para reconhecimento de palavras ditas em ambientes não controlados.


\section{Metodologia}

\quad A metodologia adotada para a realização deste trabalho consiste nos seguintes passos:

\begin{itemize}
\item Pesquisa em livros, sites, artigos e notas de aula sobre o tema abordado e seus diversos aspectos;
\item Estudo de algoritmos aplicados ao reconhecimento de fala;
\item Implementação computacional de algoritmos aplicados ao reconhecimento de fala;
\item Testes e validação dos algoritmos implementados;
\item Análise e validação dos resultados obtidos com os métodos implementados;
\item Documentação do trabalho. 

\end{itemize}

\quad No capítulo \ref{chap:referencial_teorico} é feita uma explicação do que é relevante para este trabalho com base na literatura.

























