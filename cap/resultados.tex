\chapter{RESULTADOS}
\thispagestyle{plain}
\label{chap:anal}

\quad Foram estudados e avaliados três diferentes abordagens para o reconhecimento de palavras, sempre levando em consideração o contexto da aplicação e caso de teste. 
O trabalho foi dividido em duas etapas na primeira etapa ocupou-se da captura e tratamento do sinal e extração de características. \\

\quad Foram avaliados um método estocástico (HMM), um método determinístico (correlação entre vetores) e um método de inteligência articial (SOM). 
\begin{table}[H]
\centering
\caption{Taxa de acertos do método determinístico}
\label{tab:comp}
\smallskip
\begin{tabular}{|l|l|l|l|}
\hline
 Palavra & Ambiente Silencioso & Ambiente Ruidoso & Média\\[0.5ex]
\hline
&&&\\[-2ex]
Ajuda & 70\% &  60\% &  65\%\\[0.5ex]
\hline
&&&\\[-2ex]
Assalto & 80\% &  40\% &  60\% \\[0.5ex]
\hline
&&&\\[-2ex]
Ladrão & 60\% & 34\% & 47\%\\[0.5ex]
\hline
&&&\\[-2ex]
Polícia & 90\% & 60\% & 75\%\\[0.5ex]
\hline
&&&\\[-2ex]
Socorro & 90\% & 80\% & 85\%\\[0.5ex]
\hline
&&&\\[-2ex]
Média & 78\% & 54,8\% & 66,4\%\\[0.5ex]
\hline
\end{tabular}
\end{table}

A tabela \ref{tab:comp} mostra a porcentagem de acerto para cada palavra usando o método determinístico. Cada palavra foi inserida dez vezes por um número aleatório de pessoas  em um ambiente com pouco ruído a média da taxa de acertos foi de 78\% e em um ambiente com muito ruído esta taxa caiu para 66,4\%.

\begin{table}[H]
\centering
\caption{Taxa de acertos do método determinístico}
\label{tab:comp2}
\smallskip
\begin{tabular}{|l|l|l|l|}
\hline
 Palavra & Isolado & Contínuo & Média\\[0.5ex]
\hline
&&&\\[-2ex]
Ajuda & 60\% &  35\% &  47,5\%\\[0.5ex]
\hline
&&&\\[-2ex]
Assalto & 40\% &  35\% &  37,5\% \\[0.5ex]
\hline
&&&\\[-2ex]
Ladrão & 30\% & 34\% & 32\%\\[0.5ex]
\hline
&&&\\[-2ex]
Polícia & 60\% & 37\% & 48,5\%\\[0.5ex]
\hline
&&&\\[-2ex]
Socorro & 80\% & 40\% & 60\%\\[0.5ex]
\hline
&&&\\[-2ex]
Média & 59\% & 36,6\% & 45,1\%\\[0.5ex]
\hline
\end{tabular}
\end{table}


A tabela \ref{tab:comp2} mostra a porcentagem de acerto para cada palavra usando o método determinístico com abordagem contínua. Cada palavra foi inserida vinte vezes por um número aleatório de pessoas  em um ambiente ruidoso. A média da taxa de acertos foi de 45,1\% .

\begin{table}[H]
\centering
\caption{Taxa de acertos do algoritmo de Kohonen}
\label{tab:comp3}
\smallskip
\begin{tabular}{|l|l|l|l|}
\hline
 Palavra & Ambiente Silencioso & Ambiente Ruidoso & Média\\[0.5ex]
\hline
&&&\\[-2ex]
Ajuda & 70\% &  65 \% &  67,5\% \\[0.5ex]
\hline
&&&\\[-2ex]
Assalto & 80\% &  55 \% &  67,5\% \\[0.5ex]
\hline
&&&\\[-2ex]
Ladrão & 75\% &  65 \%&  70\% \\[0.5ex]
\hline
&&&\\[-2ex]
Polícia & 75 \% & 65\% & 70\% \\[0.5ex]
\hline
&&&\\[-2ex]
Socorro & 80\% & 70\% & 75\% \\[0.5ex]
\hline
&&&\\[-2ex]
Média & 76\% & 64\% & 70\%\\[0.5ex]
\hline
\end{tabular}
\end{table}

A tabela \ref{tab:comp3} mostra  a porcentagem de acerto para cada palavra usando o algoritmo SOM. Cada palavra foi inserida dez vezes por um número aleatório de pessoas  em um ambiente com pouco ruído a média da taxa de acertos foi de 76\% e em um ambiente com muito ruído esta taxa caiu para 64\%.

\begin{table}[H]
\centering
\caption{Comparação entre os métodos}
\label{tab:comp5}
\smallskip
\begin{tabular}{|l|l|l|l|}
\hline
 -- & Determinístico & HMM & SOM\\[0.5ex]
\hline
&&&\\[-2ex]
Taxa de acerto & 66,4\% &  -- &  70\%\\[0.5ex]
\hline
&&&\\[-2ex]
Tempo de execução & rápido &  viciada &  médio \\[0.5ex]
\hline
&&&\\[-2ex]
Custo computacional & baixo & alto & alto\\[0.5ex]
\hline
\end{tabular}
\end{table}

A tabela \ref{tab:comp5} traz uma comparação entre os métodos estudados. O algorimto HMM, como foi explicado no capítulo \ref{chap:hmm}, é baseado em probabilidade e nas transições de um estado a outro. Para garantir a homogeneidade do sistema
os valores de transição dos estados são iguais isso gera um vício. Sempre que uma palavra é identificada não ocorre transição de estados, retornando sempre o estado anterior.

Para o caso estudado neste trabalho podemos concluir que o  método determinístico foi o que se mostrou mais eficiente, pois este consome menos recursos computacionais e possui uma taxa de acerto satisfatória.















































 

