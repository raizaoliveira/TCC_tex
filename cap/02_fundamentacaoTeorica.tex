\chapter{FUNDAMENTAÇÃO TEÓRICA}
\label{chap:referencial_teorico}
\thispagestyle{plain}
De acordo com \cite{fundRecFala}, os  sistemas de reconhecimento de fala podem ser classificados em três grupos de acordo com a técnica utilizada. Estes grupos são :
\begin{itemize}
\item Reconhecedores por inteligência artificial;
\item Reconhecedores por comparação de padrões;
\item Reconhecedores baseados na análise acústico-fonética.
\end{itemize}
\section{Sistemas de reconhecimento de fala }

\label{sec:sistemasdereconhecimentodefala}

\subsection{Reconhecedores por comparação de padrões}

Estes reconhecedores usam o príncipio de que o sistema foi treinado para reconhecer os padrões. Os sistemas por reconhecimento de padrões possuem duas fases diferentes :
\begin{itemize}
\item Treinamento;
\item Reconhecimento.
\end{itemize}

Durante a fase de treinamneto são criados padrões de referência para o sitema. Na fase de reconhecimento compara-se os padrões obtidos com os padrões de referência criados na fase anterior e calcula-se uma medida de similaridade entre os padões. O padrão mais similar ao desconheido é escolhido como reconhecido. Os sistemas que se baseiam nos Modelos Ocultos de Markov (HMM) se encaixam nesta categoria.\\

Dentre as diversas razões para usar a abordagem de comparação de padrões para reconheimento de fala podemos citar a simplicidade de uso, por ser um método de fácil entendimento que possui uma rica fundamentação matemática e é amplamente utilizado,  e a robustez, trata-se de um método robusto e invariante para diferentes vocábularios, algoritmos de comparação de padrão e regras de decisão. Isto torna esta abordagem apropriada para uma vasta gama de unidades de fala, como fonemas, palavras isoladas ou frases  \cite{fundRecFala}. 

\subsection{Reconhecedores baseados na análise acústico-fonética}

Os sistemas baseados na análise acústico-fonética decodificam o sinal de fala  baseados nas características acústicas deste sinal e na relação entre elas \cite{kluwer}. Os sistemas de análise desta classe devem considerar propriedades acústicas invariantes. Entre estas características estão a classificação entre sonoro e não sonoro, segmentção do sinal da fala, detecção das características que descrevem as unidades fonéticas e escolha do padrão que mais corresponde à sequencia de unidades fonéticas.\\

Os reconhecedores baseados na análise acústico-fonética trabalham em duas etapas. O primeiro passo na análise acústico fonética é chamado de fase de segmentação e rotulagem \cite{fundRecFala}. Este passo envolve a segmentação do sinal da fala em regiões discretas, no tempo, onde as propriedades acústicas do sinal são representadas por um único fonema, ou estado. Em seguida uma ou mais etiqueta fonética é associada a cada região segmentada de acordo com as propriedades acústicas. O segundo passo para o reconhecimento tenta determinar uma palavra válida a partir da sequência de etiquetas fonéticas obtidas na fase anterior. As palavras são obtidas a partir de um determinado vocabulário, as palavras obtidas fazem sentido sintático e tem significado semântico.

\subsection{Reconhecedores baseados em inteligência artificial}

Os sistemas de reconhecimento de fala que utilizam a inteligência artificial usa propriedades tanto dos reconhecedores por comparação de padrões quanto dos reconhecedores baseados na análise acústico-fonética. Sistemas com redes neurais são encaixados nesta classe. As redes Multilayer Perceptron usam uma matriz de ponderação que representa as conexões entre os nós da rede, e cada saída esta associada a uma unidade a ser reconhecida \cite{kluwerNeural}.

A abordagem de inteligência artificial se baseia no processo humano natural de ouvir, analisar e tomar uma decisão sobre as características acústicas medidas para reconher a fala. Faz parte do processo de reconheciemento de fala pela abordagem de inteligência artificial o processo de segmentação e rotulagem usado na análise acústico-fonética  \cite{fundRecFala}. Esta abordagem aplica o conceito de que o conhecimento é dinâmico  e os modelos devem adaptar-se frequentemente. 



%alguns exemplos de citação:\\
%\cite{berquo1980fatores}\\
%\cite{santos1980dinamica}\\
%\cite{NBR6023:2002}\\
%\cite{NBR14724:2005}\\
%\cite{NBR10520:2002}\\
%\cite{lessa2004manual}\\
%\cite{rey2000planejar}\\
%\cite{rajagopalan2003identidade}\\
%\cite{flemming1999calculo}\\
%\cite{gonccalves2}\\
%\cite{salgado2002nutriccao}